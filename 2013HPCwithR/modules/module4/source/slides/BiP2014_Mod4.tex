%	Use the standard preamble for Beamer slides of all 
%		statsTeachR modules
%		(\input, not \include, as \include can't access 
%		things in higher-level directories since it needs 
%		write permission there, which it doesn't have; and 
%		in some settings the preamble may be in a higher-level
%		directory than the source file.)
%	This path assumes the preamble is in the parent directory,
%		modify this if that changes.
\input{standard_beamer_preamble}

%	The following variables are assumed by the standard preamble:
%	Global variable containing module name:
\title{BiP 2014: Module 4}
%	Global variable containing module shortname:
%		(Currently unused, may be used in future.)
\newcommand{\ModuleShortname}{shortName}
%	Global variable containing author name:
\author{Andrea S. Foulkes}
%	Global variable containing text of license terms:
\newcommand{\LicenseText}{Made available under the Creative Commons Attribution-ShareAlike 3.0 Unported License: http://creativecommons.org/licenses/by-sa/3.0/deed.en\textunderscore US }
%	Instructor: optional, can leave blank.
%		Recommended format: {Instructor: Jane Doe}
\newcommand{\Instructor}{}
%	Course: optional, can leave blank.
%		Recommended format: {Course: Biostatistics 101}
\newcommand{\Course}{}

\usepackage{multicol}


%	******	Document body begins here	**********************

\begin{document}

%	Title page
\begin{frame}[plain]
	\titlepage


\end{frame}

%	******	Everything through the above line must be placed at
%		the top of any TeX file using the statsTeachR standard
%		beamer preamble. 





\begin{frame}{Genome Wide Association Studies (GWAS)}

\emph{The overarching \underline{\bf goal} of genome wide association studies is to identify genes associated with complex traits.}

\bigskip
Three broadly-defined data components:
\begin{enumerate}
\item Genetic information
\item Trait (phenotype) measuring disease progression or status
\item Demographic and clinical covariates 
\end{enumerate}
\begin{figure}[ht]
\fbox{\includegraphics[width=0.5\textwidth]{Pictures/covariate.pdf}}
\end{figure}
\end{frame}

\begin{frame}{Data components and terminology}
Genetic information
\begin{itemize}
\item Humans carry 2 \emph{homologous chromosomes}:
	\begin{itemize}
	\item segments of DNA, one inherited from each parent.
	\item code for same trait, may carry different genetic information.
	\end{itemize}
\item \emph{Nucleotide}:
	\begin{itemize}
	\item DNA base + sugar molecule + phosphate.
	\item used interchangeably with \emph{base}.
	\end{itemize}
\item Gene:
	\begin{itemize}
	\item region of DNA
	\item code for proteins or involved in regulation of production of proteins from other segments of DNA
	\end{itemize}
\end{itemize}
\end{frame}
	
\begin{frame}{Data components and terminology}
\begin{multicols}{2}
\centerline{\includegraphics[scale=.6,angle=0]{polymorphism2.jpg}}
\vspace{.25in}
\begin{itemize}
\item SNP ($x$): basic unit of analysis, typically coded 0, 1, 2 for number of variant alleles on 2 chromosomes
\item Trait ($y$): measure of disease progression or disease status.
\end{itemize}
\end{multicols}
\scriptsize
\underline{Definitions}: 
\begin{enumerate}
\item[-] {\bf polymorphism}: genetic variant occurring in greater than $1\%$ of a population
\item[-] {\bf single nucleotide polymorphism (SNP)}: variant at a single site (base pair position) on the genome.
\end{enumerate}

\end{frame}

\begin{frame}
\fbox{\includegraphics[angle=0,width=.9\textwidth]{Pictures/SNPtoAA.jpg}}
\end{frame}


\begin{frame}{Data components and terminology}
\fbox{\includegraphics[angle=0,width=1\textwidth]{ldblocks.pdf}}
\end{frame}

\begin{frame}{Data components and terminology}
\fbox{\includegraphics[angle=0,width=1\textwidth]{Pictures/markers_inherit.pdf}}
\end{frame}

\begin{frame}{Data components and terminology}
\fbox{\includegraphics[angle=0,width=1\textwidth]{Pictures/markers_inherit2.pdf}}
\end{frame}

\begin{frame}{Data components and terminology}
Trait
\begin{itemize}
\item clinical outcome or phenotype, measured \emph{in vivo} or \emph{in vitro}.
\item {\bf quantitative}, {\bf binary} (diseased or not diseased), survival (censored), longitudinal/multivariate.
\item e.g. total cholesterol, triglyceride levels, heart attack, CD4+ cell count, viral load, AIDS defining event, time to death, repeated measures of total cholesterol, etc.
\end{itemize}
Covariates
\begin{itemize}
\item environmental, clinical and demographic data.
\item potential predictors, confounders, effect modifiers, effect mediators (causal pathway variables).
\item also referred to as \emph{predictors}, \emph{confounders}, \emph{explanatory variables}, \emph{independent variables}.
\item e.g. age, gender, race/ethnicity, BMI, smoking status, etc.
\end{itemize}
\end{frame}


\begin{frame}{GWAS Analysis}
``Typical" analysis approach:
\begin{itemize}
\item Separate test of association (based on multivariable linear model) for each SNP $\rightarrow$ p-value for each SNP.

\begin{equation*}
Y=X\beta + Z \gamma + \epsilon
\end{equation*}

\begin{equation*}
H_0: \beta=0
\end{equation*}

\item Adjust to control Family Wise Error Rate (FWER) in context of multiple testing: 

\begin{equation*}
FWER=Pr(\text{reject at least one }H_0^k \; | \text{all } H_0^k \text{ are true})
\end{equation*}

\item Typically control at level $\alpha=0.05$ using Bonferonni adjustment $\rightarrow$ P-value statistically significant if less than $0.05/1,000,000=-5 \times 10^{-8}$.
\end{itemize}
\end{frame}

\begin{frame}{CARDIoGRAM summary level data}

{\bf C}oronary {\bf AR}tery {\bf DI}sease {\bf G}enome-wide {\bf R}eplication {\bf A}nd {\bf M}eta-anaylsis (CARDIoGRAM) data:

\begin{itemize}
\item Meta-analysis of 14 GWAS of coronary artery disease (CAD): 22,233 cases and 64,762 controls
\item Replication study in additional $56,682$ individuals
\item Available data (after pre-processing): p-values for $965,220$ SNPs in $19,216$ genes.  
\end{itemize}
\end{frame}

\begin{frame}
\includegraphics[angle=0,width=1\textwidth]{ng784-F1.jpg}

\scriptsize Schunkert et al. Nature Genetics 43,  333-338 (2011) doi:10.1038/ng.784
    
\end{frame}

\begin{frame}{Lab Assignment}
\begin{enumerate}
\item Conduct a simulation study to identify an appropriate p-value threshold for statistical significance (assuming independence of SNPs):
\begin{itemize}
	\item generate $965,220$ p-values from a uniform distribution
	\item determine value corresponding to the $5$th percentile
	\item repeat $500$ times and record $5$th percentile of this distn.
	\end{itemize}
\item Repeat (1) while accounting for within gene correlation
	\begin{itemize}
	\item assume inverse normally transformed p-values ($p_{ij}$) arise from a random effects model (i indicates gene and j indicates SNP):
	
	\begin{equation*}
	y_{ij}= b_i +\epsilon_{ij}
	\end{equation*}
	
	$p_{ij}=\Phi^{-1}(y_{ij})$, $b_i \sim N(0,0.4)$, $\epsilon_{ij} \sim N(0,1)$ and $b_i \perp \epsilon_{ij}$.
	 
	\end{itemize}
\item Repeat (2) where the random gene level effects arise from a $N(0,\sigma_b^2)$ and $\sigma_b^2$ ranges from 0.2 to 1.2 in increments of 0.2. 
\end{enumerate} 
\end{frame}

\end{document}