%	Use the standard preamble for Beamer slides of all 
%		statsTeachR modules
%		(\input, not \include, as \include can't access 
%		things in higher-level directories since it needs 
%		write permission there, which it doesn't have; and 
%		in some settings the preamble may be in a higher-level
%		directory than the source file.)
%	This path assumes the preamble is in the parent directory,
%		modify this if that changes.
\input{../standard_beamer_preamble}

%	The following variables are assumed by the standard preamble:
%	Global variable containing module name:
\title{Simulation and parallelization in R}
%	Global variable containing module shortname:
%		(Currently unused, may be used in future.)
\newcommand{\ModuleShortname}{simPar}
%	Global variable containing author name:
\author{Nicholas G Reich}
%	Global variable containing text of license terms:
\newcommand{\LicenseText}{Made available under the Creative Commons Attribution-ShareAlike 3.0 Unported License: http://creativecommons.org/licenses/by-sa/3.0/deed.en\textunderscore US }
%	Instructor: optional, can leave blank.
%		Recommended format: {Instructor: Jane Doe}
\newcommand{\Instructor}{}
%	Course: optional, can leave blank.
%		Recommended format: {Course: Biostatistics 101}
\newcommand{\Course}{}



%	******	Document body begins here	**********************

\begin{document}

%	Title page
\begin{frame}[plain]
	\titlepage
\end{frame}

%	******	Everything through the above line must be placed at
%		the top of any TeX file using the statsTeachR standard
%		beamer preamble. 


\begin{frame}{Module learning goals}

	\begin{block}{At the end of this module you should be able to...}
		

		\begin{itemize}

			\item{Simulate data from a parametric distribution.}

			\item{Formulate a statistical model and simulate data from it.}

			\item{Design and implement a simulation experiment to test a hypothesis.}
                        
                        \item{Run simulations in parallel, when appropriate.}
                        
		\end{itemize}

	\end{block}

\end{frame}



\begin{frame}{What is simulation?}



        \begin{block}{Definitions}
		

		\begin{itemize}

			\item{Broadly: ``The technique of imitating the behaviour of some situation or process (whether economic, military, mechanical, etc.) by means of a suitably analogous situation or apparatus, esp. for the purpose of study or personnel training.'' (from the {\em OED})}

			\item{In science: Creating a model that imitates a physical or biological process.}

        		\item{In statistics: The generation of data from a model using rules of probability.}
                                                
		\end{itemize}

	\end{block}

\end{frame}



\begin{frame}{What simulations have you run?}


        \begin{itemize}

                \item Drawing pseudo-random numbers from a probability distribution (e.g. proposal distributions, ...).
                
                \item Generating data from a specified model (e.g. building a template dataset, calculating statistical power).
                
                \item Resampling existing data (e.g. permutation, bootstrap).

        \end{itemize}

\vskip2em

\begin{block}{In the right setting, any of the above methods can be used in a data analysis.}
\end{block}
\end{frame}

\begin{frame}{Key simulation functions in R}


\begin{block}{{\tt rnorm()}, {\tt rpois()}, etc...}

Built-in functions for simulating data from known parametric distributions.

\end{block}


\begin{block}{{\tt sample()}}

Base R function for sampling data (with or without replacement).



\end{block}


\end{frame}

\end{document}